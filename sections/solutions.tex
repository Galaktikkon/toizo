\section*{Rozwiązania}


\begin{solution}{nawiasy}
Tekst rozwiązania...
\end{solution}

\begin{solution}{programowanie}
Tekst rozwiązania...
\end{solution}

\begin{solution}{własności RE}
Tekst rozwiązania...
\end{solution}

\begin{solution}{definicja rozstrzygalności}
Tekst rozwiązania...
\end{solution}

\begin{solution}{równoważność modeli obliczeń}
Tekst rozwiązania...
\end{solution}

\begin{solution}{dowód, że przecięciem RE i coRE jest R}
Tekst rozwiązania...
\end{solution}

\begin{solution}{twierdzenia dotyczące enumeratorów}
Tekst rozwiązania...
\end{solution}

\begin{solution}{separacja języków z core}
Tekst rozwiązania...
\end{solution}

\begin{solution}{język prefiksowy}
Tekst rozwiązania...
\end{solution}

\begin{solution}{redukcje}
Tekst rozwiązania...
\end{solution}

\begin{solution}{hierarchia arytmetyczna}
Tekst rozwiązania...
\end{solution}

\begin{solution}{hierarchia arytmetyczna - niepełny co do długości}
Tekst rozwiązania...
\end{solution}

\begin{solution}{hierarchia arytmetyczna - sygnatura}
Tekst rozwiązania...
\end{solution}

\begin{solution}{hierarchia arytmetyczna - język rzadki}
Tekst rozwiązania...
\end{solution}

\begin{solution}{hierarchia arytmetyczna - enumeratory}
Tekst rozwiązania...
\end{solution}

\begin{solution}{enumerator z wycofywaniem}
Tekst rozwiązania...
\end{solution}

\begin{solution}{wielomianowe warianty SAT-a}
Tekst rozwiązania...
\end{solution}

\begin{solution}{NP-trudne warianty SAT-a}
Tekst rozwiązania...
\end{solution}

\begin{solution}{przynależność CLIQUE do NP}
Tekst rozwiązania...
\end{solution}

\begin{solution}{przynależność HAMILTONIAN-CYCLE do NP}
Tekst rozwiązania...
\end{solution}


\begin{solution}{zamkniętość klas}
Tekst rozwiązania...
\end{solution}

